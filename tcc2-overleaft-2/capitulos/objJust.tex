%\section{Objetivos e Justificativas}%\label{cap:objJust}
% ---

%-
\section{Objetivo Geral}
%-
O objetivo geral deste trabalho é desenvolver o protótipo de um VANT controlado por um sistema embarcado com visão computacional que será implementado em um computador complementar (Raspberry Pi), o qual se comunicará com a controladora de voo por protocolo de comunicação.

A ideia é desenvolver um sistema de robótica que atue em um VANT através de controle e automação, tendo como objetivo, controlar e movimentar o VANT usando visão computacional. Para isso, será implementado um algoritmo de visão computacional para reconhecimento facial que será pré treinado utilizando CNN (Convolucional Neural Network), o qual identificará faces que constam no conjunto de dados treinado pela rede neural. Além disso, um novo sistema de coordenadas será construído baseando-se no sistema de coordenadas do OpenCV, o qual irá gerar um plano de coordenas do tipo cartesianas. As componentes X e Y adquiridas pelo sistema cartesiano serão enviadas para a controladora de voo fazendo com que o VANT se desloque de acordo com a movimentação percebida pelo sistema de visão computacional.  

A validação do sistema será através da abstração do hardware por um conjunto de ferramentas de software que simulam o funcionamento do VANT.   

%-
%%%%%%%%%%%%%%%%%%%%%%%%%%%%%%%%%%%%%%%%%%%%%%%%%%%%%%%%%%
%Normalmente não há problemas em usar caracteres acentuados em arquivos
%bibliográficos (\texttt{*.bib}). Porém, como as regras da ABNT fazem uso quase
%abusivo da conversão para letras maiúsculas, é preciso observar o modo como se
%escreve os nomes dos autores. Na~\autoref{tabela-acentos} você encontra alguns
%exemplos das conversões mais importantes. Preste atenção especial para `ç' e `í'
%que devem estar envoltos em chaves. A regra geral é sempre usar a acentuação
%neste modo quando houver conversão para letras maiúsculas.

%\begin{table}4htbp]
%\caption{Tabela de conversão de acentuação.}
%\label{tabela-acentos}

%\begin{center}
%\begin{tabular}{ll}\hline\hline
%acento & \textsf{bibtex}\\
%à á ã & \verb+\`a+ \verb+\'a+ \verb+\~a+\\
%í & \verb+{\'\i}+\\
%ç & \verb+{\c c}+\\
%\hline\hline
%\end{tabular}
%\end{center}
%\end{table}
%%%%%%%%%%%%%%%%%%%%%%%%%%%%%%%%%%%%%%%%%%%%%%%%%%%%%%%%%%%%%%%

\section{Objetivos Específicos}
\begin{itemize}
	\item Desenvolver um algoritmo de visão computacional para reconhecimento facial;
	\item Desenvolver um sistema de controle e automação para VANT baseando-se em Visão Computacional;
	\item Comparar o desempenho do algoritmo de visão computacional entre o computador utilizado nas simulações e o computador complementar (Raspberry Pi);
	\item Realizar testes e adquirir dados para determinar se é viável ou não utilizar visão computacional para definir um sistema de robótica para um VANT.
\end{itemize}

\section{Resultados Esperados}

\begin{itemize}
	\item Reconhecimento de faces inseridas no conjunto de dados pré-treinado;
	\item Distinguir entre faces inseridas no conjunto de dados e as que não foram inseridas;
	\item Desenvolver um sistema que possibilite guiar o VANT;
	\item Demonstrar que é possível guiar o VANT em direções estipuladas pelo sistema de visão computacional.
	
\end{itemize}

%%%%%%%%%%%%%%%%%%%%%%%%%%%%%%%%%%%%%%%%%%%%%%%%%%%%%%%%%%%%%%%%%%%
%Consulte a FAQ com perguntas frequentes e comuns no portal do \abnTeX:
%\url{https://code.google.com/p/abntex2/wiki/FAQ}.

%Inscreva-se no grupo de usuários \LaTeX:
%\url{http://groups.google.com/group/latex-br}, tire suas dúvidas e ajude
%outros usuários.
%%%%%%%%%%%%%%%%%%%%%%%%%%%%%%%%%%%%%%%%%%%%%%%%%%%%%%%%%%%%%%%%%%%

\section{Visão Geral da Metodologia}

A metodologia utilizada pode ser definida em 3 etapas que são exemplificas de maneira procedural e ilustrada na figura \ref{fig:diagmetpesq}. 

As etapas foram definidas da seguinte forma:

\begin{itemize}
	\item Uma primeira etapa de pesquisa bibliográfica buscando definir quais materiais encontrados seriam utilizados como referência para prosseguir, e através deles, verificar se era possível chegar a um resultado satisfatório e se o protótipo teria uma aplicação prática e justificável. 
	
	\item Depois das pesquisas, veio a etapa que definiu quais bibliotecas, ferramentas, linguagem de programação, protocolos e qual seria a maneira de simulação e validação a ser utilizada, assim como as instalações e configurações dos sistemas para posterior codificação dos algoritmos e testes de bibliotecas e códigos.
	
	\item A ultima etapa foi para realizar simulações de funcionamento e comportamento para se chagar em resultados.
\end{itemize}

\begin{figure}[H]
	\centering	
	\caption{Metodologia utilizada}
	\fontsize{9pt}{12pt}\selectfont
	%\color{white}
	\def\svgwidth{15cm}
	\input{figs/svg/diagmetpesq.pdf_tex}
	\legend{Fonte: do autor.}
	\label{fig:diagmetpesq}
\end{figure}

Como o foco principal do trabalho é o controle de um VANT utilizando visão computacional, primeiramente se realizou uma pesquisa para descobrir se existe um computador complementar que pudesse comportar um software de visão computacional e que também tivesse capacidade computacional de compilar e rodar o algoritmo e com processamento suficiente para processar a tecnologia de visão computacional. A pesquisa também abordou como comunicar o computador complementar com a controladora do VANT e se existia a possibilidade de enviar comandos para a controladora de voo. 

Foi preciso buscar na internet muito material, principalmente, tutoriais para aprender e dominar o funcionamento das ferramentas que seriam utilizadas no desenvolvimento, por exemplo, o OpenCV e as ferramentas de desenvolvimento para VANTs. Essas tecnologias são muito vastas e possuem muitas funcionalidades e módulos. Desta forma, foi preciso testar quais seriam as melhores maneiras de utiliza-las no projeto, e quais módulos utilizar. Logo foi gasto muito tempo simulando e testando esses módulos para adequar quais seriam utilizados e serviriam na implementação.  

Uma boa parte do tempo de pesquisa  do trabalho foi buscar bibliografias sobre IA e Deep Learning, entender sobre redes neurais e visão computacional para o entendimento de como funcionam essas tecnologias.

\section{Justificativa} 
%-
Devido ao aumento da criminalidade e dos altos índices de violência no cotidiano das pessoas, surgiu a cultura do medo e o sentimento de insegurança. Tais fatores demandaram algumas mudanças nos serviços de segurança patrimonial, bem como, nas formas de monitoramento. Nesse sentido, tornou-se necessário expandir as formas de controle, seja por meio de câmeras de vigilância ou monitoramento eletrônico \cite{quatro}. No entanto, os equipamentos atuais como as câmeras utilizadas na segurança já não são mais eficientes e possuem tecnologias ultrapassadas. 

Na cidade do Rio de janeiro existe um sistema de câmeras de alta tecnologia com capacidade para realizar reconhecimento facial. Este sistema foi implantado para testes através de uma parceria entre a OI e a Huawei \cite{cinco}, sendo que as câmeras e a tecnologia de reconhecimento facial embarcados são de propriedade da Chinesa Huawei e possuem alto custo financeiro, o que contrasta com a atual realidade de e alguns estados Brasileiros que passam por uma crise financeira só que  precisam investir em segurança \cite{seis}.

Este contexto de insegurança pública e falta de investimento nacional em tecnologias de segurança é que originou a motivação para a proposta deste trabalho, ou seja, a necessidade de investir em tecnologias de baixo ou médio custo, que possam ser aplicados para melhorar o desempenho do sistema de segurança no Brasil, visando a vigilância, através do sensoriamento utilizando câmeras e hardware de custo acessível e softwares livres.
%-

\section{Organização do Documento}

O capitulo dois contém a fundamentação teórica apresentando os conceitos de IA\footnote{Inteligência Artificial}, ML\footnote{Machine Learning} e DE\footnote{Deep Learning}, prosseguindo com as ferramentas de visão computacional e tecnologias para desenvolvimento de VANT, sendo concluída com os conceitos e teorias de voo sobre GPS na superfície terrestre. 

O capitulo três apresenta a metodologia utilizada no protótipo, a qual inclui diagramas de UMLs sobre o funcionamento do \textit{software}, e a teoria utilizada para chegar no sistema de controle e automação.

O quarto e último capítulo apresenta os resultados dos testes realizados nos simuladores e a comparação dos \textit{hardwares} no \textit{benchmarking}.

Por fim o apêndice contém os códigos fonte que foram implementados no protótipo.


