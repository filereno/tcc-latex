% ---
% RESUMOS
% ---

% RESUMO em português
\setlength{\absparsep}{18pt} % ajusta o espaçamento dos parágrafos do resumo
\begin{resumo}
A utilização de tecnologias de vanguarda como visão computacional e tecnologias de VANT está sendo muito aplicada na área de segurança ao redor do mundo, porém se faz necessário muito investimento na aquisição destas tecnologias. Uma solução é o investimento em pesquisa e desenvolvimento levando em consideração o fato de que a maioria dessas tecnologias é de código aberto, ou seja, a maior parte do investimento seria de cunho intelectual. Com isso surgiu a ideia de implementar e integrar duas das tecnologias de ponta dentro deste contexto: visão computacional e tecnologias de desenvolvimento para VANTs. Este trabalho abordou o desenvolvimento de um sistema de controle e automação para VANT mostrando que é possível o seu desenvolvimento e sua utilização na área de segurança, principalmente segurança pública. Foram demonstrados dois métodos que podem ser utilizados como controle de robótica para um VANT: regiões de interesse e sistema cartesiano. Além disso, a técnica de visão computacional utilizada foi a de reconhecimento facial. Como resultado, foi possível construir um equipamento de voo que consegue identificar um indivíduo e posteriormente se deslocar seguindo-o com o sistema de controle desenvolvido.

 \textbf{Palavras-chaves}: Visão computacional, Reconhecimento facial, Veiculo aéreo não tripulado.
\end{resumo}

% ABSTRACT in english
\begin{resumo}[Abstract]
 \begin{otherlanguage*}{english}
   %The use of avant-garde technologies such as computer vision and UAV technologies is being used a lot in the security area around the world, however it is necessary a lot of investment in the acquisition of these. One solution is to invest in research and development taking into account the fact that most of these technologies are open source, that is, most of the investment would be of an intellectual nature. With that, the idea of implementing and integrating two of the cutting edge technologies, computer vision and development technologies for UAVs, came up. The development of a control and automation system for UAV will be addressed, showing that it is possible to develop and use it in the security area, mainly public security. We will demonstrate two methods that can be used as robotics control for a UAV, the regions of interest and the Cartesian system, and the computer vision technique used will be facial recognition. Soon we will have as a result a flight equipment that can identify an individual and later move around following it with the developed control system.
   
   The use of avant-garde technologies such as computer vision and UAV technologies is being widely used in the security area around the world, however much investment is required in the acquisition of these technologies. One solution is to invest in research and development taking into account the fact that most of these technologies are open source, that is, most of the investment would be of an intellectual nature. With that in mind, came the idea of implementing and integrating two of the cutting edge technologies within this context: computer vision and development technologies for UAVs. This work addressed the development of a control and automation system for UAVs showing that it is possible to develop and use it in the security area, mainly public security. Two methods have been demonstrated that can be used as robotics control for a UAV: regions of interest and cartesian system. In addition, the computer vision technique used was facial recognition. As a result, it was possible to build a flight equipment that can identify an individual and later move around following it with the developed control system.

   \vspace{\onelineskip}
 
   \noindent 
   \textbf{Keywords}: Computer vision, Facial recognition, Unmanned aerial vehicle
 \end{otherlanguage*}
\end{resumo}