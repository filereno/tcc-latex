% ----------------------------------------------------------
% Introdução 
% Capítulo sem numeração, mas presente no Sumário
% ----------------------------------------------------------

\chapter*[Introdução]{Introdução}
\addcontentsline{toc}{chapter}{Introdução}

%Este documento segue as normas estabelecidas pela~\citeonline[3.1-3.2]{NBR6028:2003}.
A segurança pública hoje é um tema muito noticiado em veículos de mídia. Segundo estudos, hoje no Brasil aproximadamente 10\% dos crimes cometidos são solucionados \cite{um} e isso se dá principalmente devido ao fato das polícias terem um ‘’’baixo efetivo não dando conta de cobrirem certas áreas. Outro fator muito importante é a falta de investimento em tecnologias o que não acontece em países de primeiro mundo que possuem uma alta taxa de soluções de casos criminais pelo fato destes países terem um alto investimento tecnológico. Um bom exemplo seria a China que possui hoje o melhor sistema de vigilância existente, composto de um complexo sistema de reconhecimento facial, o que faz com que eles possuam uma taxa de crimes mais baixas que a de países da Europa e equivalente ao de países como a Suíça \cite{dois}.  
No Brasil, os crimes são solucionados, pelo fato da polícia geralmente adquirir provas através de alguma imagem feita por uma câmera de terceiro, ou seja, equipamentos privados que foram instalados em residências ou empresas [16]. Contudo, percebemos que no Brasil existe uma grande falta de investimentos em pesquisa focada em novas tecnologias que poderiam suprir o mercado e principalmente a área de segurança. 
Hoje existem tecnologias de visão computacional com distribuição livre que podem ser utilizadas por exemplo no mercado Brasileiro na área de segurança. Um exemplo destas tecnologias são as ferramentas de visão computacional open source, um exemplo é o OpenCV, ferramenta muito poderosa que é utilizada hoje em vários países com foco em reconhecimento facial, identificação de padrões de comportamento, detecção de objetos, entre outras várias funcionalidades, todas focadas em visão computacional e utilizando aprendizagem de máquina (Machine Learning) [17].  
Com investimento em pesquisa e desenvolvimento, a tecnologia de visão computacional focada em detecção facial e padrões de comportamento poderia ser aplicada como ferramenta para suprir a demanda de tecnologias na área de segurança pública no Brasil. Poderia ser desenvolvido um sistema integrado entre os estados utilizando um banco de dados compartilhado para detecção de possíveis criminosos, fugitivos, suspeitos com posse de armamento em meio à multidão, assim coibindo possíveis crimes. 
Alguns países que utilizam detecção facial já estão desenvolvendo sistemas integrados desta tecnologia com a utilização de um veículo aéreo não tripulado (Vant) \cite{dois} [2]. Com um drone é possível cobrir uma grande área utilizando o espaço aéreo e em casos de desaparecimentos, este equipamento poderá acessar áreas de difícil acesso. Um bom exemplo e o caso do rompimento da barragem de Brumadinho aonde ocorreu de vários corpos não serem localizados pelo fato da lama dificultar as buscas, e nesse caso, uma empresa estrangeira veio com uma solução utilizando drones que possibilitaram a busca em certos locais, drones esses que possuíam câmeras com tecnologias que possibilitavam identificar um ser vivo ou um corpo a certas profundidades na lama [18].  
Tecnologias de drones para visão computacional ainda estão em desenvolvimento e possuem algumas limitações como a comunicação de uma controladora de vôo com um computador que possa sustentar a tarefa de processar todo o sistema de imagem e desempenhar comandos de vôo para o equipamento de vôo.
No site da Ardupilot \cite{tres}[3], já existe um projeto em desenvolvimento para conectar e configurar um mini computador de baixo custo, como o Nvidia TX2, Intel Edison e a Raspberry Pi. A ideia do estudo é integrar via protocolo MAVLink os dois hardwares à Raspberry Pi com um sistema Linux embarcado e integrado à ferramenta OpenCV, e assim possibilitar o desenvolvimento de um sistema com aprendizado de máquina (Machine Learning) que envie controles de direcionamento para o equipamento de voo através da comunicação entre os dos equipamentos.  

%\section*{Figuras}\label{sec:figuras}
%\addcontentsline{toc}{section}{figuras}

%As normas da~\citeonline[3.1-3.2]{NBR6028:2003} especificam que o caption da figura %deve vir abaixo da mesma.

%A Figura~\ref{fig:log} ilustra...

%\begin{figure}[htpb]
 %  \centering
 %  \includegraphics[scale=.3]{figs/logo}
 %  \caption{Breve explicação sobre a figura. Deve vir abaixo da mesma.}
 %  \label{fig:nome dado a figura}
%   \legend{Fonte: do autor}
%\end{figure}

%\section*{Tabelas}\label{sec:tabelas}
%\addcontentsline{toc}{section}{tabelas}

%A Tabela~\ref{tab:tabela} apresenta os resultados...

%\begin{table}[htpb]
   %\centering
   %\caption{Breve explicação sobre a tabela. Deve vir acima da %mesma.}\label{tab:tabela}
   %\begin{tabular}{|l|c|c|c|c|c|c|r|}
        %\hline
        %\small{XX} & \small{FF} & \small{PP} & \small{YY} & \small{Yr} & \small{xY} & %\small{Yx} & \small{ZZ} \\ \hline
               %615 &    18      &     2558   &    0,9930  &    0,9930  &    0,9930  & %   0,9930  &    0,9930  \\ \hline
               %615 &    18      &     2558   &    0,9930  &    0,9930  &    0,9930  & %   0,9930  &    0,9930  \\ \hline
               %615 &    18      &     2558   &    0,9930  &    0,9930  &    0,9930  & %   0,9930  &    0,9930  \\ \hline
               %615 &    18      &     2558   &    0,9930  &    0,9930  &    0,9930  &  %  0,9930  &    0,9930  \\ \hline
               %615 &    18      &     2558   &    0,9930  &    0,9930  &    0,9930  & %   0,9930  &    0,9930  \\ \hline
%   \end{tabular}
%\end{table}

%\section*{Motivação}\label{sec:motivacao}
%\addcontentsline{toc}{section}{Motivação}

%\lipsum[35]

