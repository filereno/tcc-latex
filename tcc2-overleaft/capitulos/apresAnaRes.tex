\chapter{Analise dos Resultados e Considerações Finais}\label{cap:apresAnaRes}




\section{Resultados}
 
\section{Considerações Finais}

Com tudo isso se chegou a conclusão de que é possível controlar um VANT usando visão computacional, porem como a maioria dos trabalhos nessa área são de caráter experimental, se decidiu criar uma bateria de simulações utilizando softwares e implementar um algoritmo que une visão computacional (OpenCV para reconhecimento facial) com tecnologia de controle para VANT (Dronekit) e checar se as reações da simulação são suficientemente satisfatórias. Logo se determinou que o VANT ir na direção pré-determinada no algoritmo é o ponto de satisfação para uma conclusão positiva do protótipo.

Apesar dos resultados não terem sido totalmente satisfatórios, é possível evoluir o produto aplicando mais testes agregando mais recursos e utilizando equipamentos mais específicos para este fim.

Em relação aos componentes usados a maioria eram do próprio autor e não foram adquiridos outros devido ao custo devidamente auto para uma pessoa física.  
 
Em relação aos métodos usados na analise, a simulação por software comprovou ser ideal para os primeiros testes, porem se fez necessário possuir um computador com uma alta capacidade de processamento, e foi utilizado o computador que o autor ja possuía. 

Alguns paradigma em relação ao desenvolvimento desse projeto:

\begin{itemize}
	\item Em relação ao VANT o principal paradigma a ser vencido é a autonomia que na atualidade fica em torno de no máximo 45 minutos de operação.
	\item Em relação a tecnologia de visão computacional o paradigma a ser vencido é o de capacidade de processamento.
\end{itemize}

Alguns aspectos que podem ser melhorados em versões futuras para tornar o projeto
mais eficiente e utilizável são:

\begin{itemize}
	\item Utilizar um equipamento com maior poder de processamento e especifico para o fim.
	\item Maior cronograma de tempo para o desenvolvimento de um sistema de simulação mais completo e eficiente.
	\item Estender as técnicas de visão computacional para melhoria do sistema de sensoriamento de pessoas.
	\item Utilizar um sistema de controle do tipo PID ou lógica Fuzzi para controlar o VANT.
	\item Adquirir os componentes físicos para testes reais.
\end{itemize}