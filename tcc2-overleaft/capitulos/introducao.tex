\chapter{Introdução}\label{cap:introducao}


%\chapter*[Introdução]{Introdução}
%\addcontentsline{toc}{chapter}{Introdução}

%Este documento segue as normas estabelecidas pela~\citeonline[3.1-3.2]{NBR6028:2003}.
A segurança pública hoje é um tema muito noticiado em veículos de mídia, segundo estudos hoje no Brasil aproximadamente 10\% dos crimes cometidos são solucionados \cite{um} e isso se dá principalmente devido ao fato das polícias terem um baixo efetivo não dando conta de cobrirem certas áreas. Outro fator muito importante é a falta de investimento em tecnologias o que não acontece em países de primeiro mundo que possuem uma alta taxa de soluções em casos criminais pelo fato destes países terem um alto investimento tecnológico na área de segurança. Um bom exemplo seria a China que possui hoje o melhor sistema de vigilância existente, composto de um complexo produto de IA utilizando visão computacional, o que faz com que eles possuam uma taxa de crimes mais baixa que a de países da Europa e equivalente ao de países como a Suíça \cite{dois}.  

No Brasil, os crimes são solucionados, pelo fato da polícia geralmente adquirir provas através de alguma imagem feita por uma câmera de terceiro, ou seja, equipamentos privados que foram instalados em residências ou empresas [16]. Contudo, percebemos que no Brasil existe uma grande falta de investimentos em pesquisa focada em novas tecnologias que poderiam suprir o mercado e principalmente a área de segurança. 

Hoje existem tecnologias com distribuição livre que podem ser utilizadas no mercado Brasileiro na área de segurança. Um exemplo são as ferramentas de visão computacional como o OpenCV, ferramenta muito poderosa que é utilizada em vários países com foco principalmente em reconhecimento facial, identificação de padrões de comportamento, detecção de objetos, entre outras várias funcionalidades, todas utilizando aprendizagem de máquina (Machine Learning) [17].  

Com investimento em pesquisa e desenvolvimento, a tecnologia de visão computacional pode ser aplicada como ferramenta para suprir a demanda de tecnologias na área de segurança pública no Brasil.

Alguns países que utilizam detecção facial já estão desenvolvendo sistemas integrados desta tecnologia com a utilização de um veículo aéreo não tripulado (VANT) \cite{dois} [2]. Com um VANT é possível cobrir uma grande área utilizando o espaço aéreo e em casos de desaparecimentos, este equipamento poderá acessar áreas de difícil acesso. Um bom exemplo e o caso do rompimento da barragem de Brumadinho aonde ocorreu de vários corpos não serem localizados pelo fato da lama dificultar as buscas, e uma empresa estrangeira veio com uma solução utilizando VANTs que possibilitaram a busca em certos locais, esses equipamentos possuíam câmeras com tecnologias que possibilitavam identificar um corpo em meio ou dentro da lama [18].  

%Tecnologias de drones para visão computacional ainda estão em desenvolvimento e possuem algumas limitações como a comunicação de uma controladora de vôo com um computador que possa sustentar a tarefa de processar todo o sistema de imagem e desempenhar comandos de vôo para o equipamento de vôo.

No site da Ardupilot \cite{tres}[3], já existe um projeto em desenvolvimento para conectar e configurar um computador complementar,  o Nvidia Jetson\footnote{\url{https://www.nvidia.com/pt-br/autonomous-machines/embedded-systems/jetson-nano/}}, Google Coral Dev Board\footnote{\url{https://coral.ai/products/dev-board}} e Raspberry Pi\footnote{\url{https://www.raspberrypi.org/products/raspberry-pi-3-model-b/}} a uma controladora de voo, essa conexão utiliza um protocolo chamado MAVLink e possui uma documentação bem desenvolvida que explica detalhadamente como controlar um VANT utilizando a ferramenta Dronekit, e a partir dele é possível prototipar uma grande diversidade de projetos unindo ferramentas e tecnologias diversas ao VANT.

Com isso surgiu a ideia do estudo de integrar via protocolo de comunicação (MAVLink) à Raspberry Pi com a controladora de voo do VANT, embarcar um sistema Linux com ferramentas de visão na Raspberry Pi, e assim possibilitar o desenvolvimento de um sistema com IA focado em visão computacional que possibilite o desenvolvimento de controles direcionais que atuem no VANT, sendo assim possível controla-lo através de um sistema que tenha uma câmera como sensor de orientação e direcionamento.  





%\section*{Figuras}\label{sec:figuras}
%\addcontentsline{toc}{section}{figuras}

%As normas da~\citeonline[3.1-3.2]{NBR6028:2003} especificam que o caption da figura %deve vir abaixo da mesma.

%A Figura~\ref{fig:log} ilustra...

%\begin{figure}[htpb]
 %  \centering
 %  \includegraphics[scale=.3]{figs/logo}
 %  \caption{Breve explicação sobre a figura. Deve vir abaixo da mesma.}
 %  \label{fig:nome dado a figura}
%   \legend{Fonte: do autor}
%\end{figure}

%\section*{Tabelas}\label{sec:tabelas}
%\addcontentsline{toc}{section}{tabelas}

%A Tabela~\ref{tab:tabela} apresenta os resultados...

%\begin{table}[htpb]
   %\centering
   %\caption{Breve explicação sobre a tabela. Deve vir acima da %mesma.}\label{tab:tabela}
   %\begin{tabular}{|l|c|c|c|c|c|c|r|}
        %\hline
        %\small{XX} & \small{FF} & \small{PP} & \small{YY} & \small{Yr} & \small{xY} & %\small{Yx} & \small{ZZ} \\ \hline
               %615 &    18      &     2558   &    0,9930  &    0,9930  &    0,9930  & %   0,9930  &    0,9930  \\ \hline
               %615 &    18      &     2558   &    0,9930  &    0,9930  &    0,9930  & %   0,9930  &    0,9930  \\ \hline
               %615 &    18      &     2558   &    0,9930  &    0,9930  &    0,9930  & %   0,9930  &    0,9930  \\ \hline
               %615 &    18      &     2558   &    0,9930  &    0,9930  &    0,9930  &  %  0,9930  &    0,9930  \\ \hline
               %615 &    18      &     2558   &    0,9930  &    0,9930  &    0,9930  & %   0,9930  &    0,9930  \\ \hline
%   \end{tabular}
%\end{table}

%\section*{Motivação}\label{sec:motivacao}
%\addcontentsline{toc}{section}{Motivação}

%\lipsum[35]

