% ---
\chapter{Objetivos e Justificativas}\label{cap:objJust}
% ---
A ideia surgiu devido ao interesse em integrar um veículo aéreo não tripulado, no caso um drone do tipo quadricóptero, com algum dispositivo para realização de alguma tarefa, e após pesquisar várias trabalhos, protótipos experimentais, bibliografia sobre o assunto e sites especializados em tecnologias aplicadas em Vants, como exemplo a visão computacional, se descobriu que é possível conectar através de um protocolo de comunicação, uma controladora de drone [7] a um computador complementar ou como é mais abordado mini computador [8], e enviar comandos para o Vant, criando um sistema de controle e automação de Vant através de visão computacional [3]. Em seguida também se descobriu que e possível integrar a ferramenta OpenCV na Raspberry Pi fazendo a integração de visão computacional com a tecnologia de um Vant.
Como o foco principal do trabalho é o controle utilizando visão computacional, primeiro se realizou uma pesquisa para descobrir se existe um computador complementar que pudesse comportar um software de visão computacional e que também tivesse capacidade computacional de compilar e rodar o algoritmo, e com processamento o suficiente para processar a tecnologia de visão computacional.
O Raspberry Pi foi a melhor opção devido a alguns fatores como; custo, tamanho, tempo.
A segunda pesquisa abordou como comunicar o computador complementar com a controladora do Vant e se existia a possibilidade de enviar comandos para a controladora de voo. 
Foi preciso buscar na internet muito material principalmente tutorias para aprender e dominar o funcionamento das ferramentas que seriam utilizadas no desenvolvimento, exemplo é o OpenCV, essa ferramenta e muito vasta possui muitas funcionalidades e módulos, então foi preciso descobrir qual seria a melhor maneira de utiliza-la no projeto, quais módulos utilizar, logo foi gasto muito tempo praticando e testando esses módulos para adequar quais seriam utilizados.  
Uma boa parte do trabalho foi realizando uma pesquisa bibliográfica sobre Deep Learning para o entendimento de como funcionam as tecnologias de visão computacional.

%-
\section{Objetivos Gerais}
%-
O objetivo geral deste trabalho é desenvolver o protótipo de um VANT controlado por um sistema embarcado com visão computacional para reconhecimento de padrões de comportamento de multidões, com reconhecimento facial, que possa ser empregado como uma ferramenta de segurança, podendo auxiliar em casos policiais, vigilância patrimonial, filmagens áreas, buscas em mata fechada, fiscalização de fronteira e aplicações civis nas quais seria inviável um ser humano trabalhar, realizando tarefas arriscadas e servindo como verdadeira ferramenta de trabalho.
Acredita-se que com o uso de imagens aéreas possa ser possível reconhecer e emitir alertas de possíveis suspeitos de crimes, e com isso buscar minimizar um pouco a falta de tecnologias que hoje são necessárias para ajudar no combate a violência. 
Um sistema de visão computacional integrado a um drone que através do desenvolvimento de um algoritmo e o treinamento de redes neurais, seja capaz de cumprir objetivos específicos, tendo como principal, o de identificar indivíduos através de reconhecimento facial, padrões de comportamento, e objetos específicos portados, e logo tomar a ação de rastrear o elemento que foi identificado através da movimentação aérea de um drone.   

%-
%%%%%%%%%%%%%%%%%%%%%%%%%%%%%%%%%%%%%%%%%%%%%%%%%%%%%%%%%%
%Normalmente não há problemas em usar caracteres acentuados em arquivos
%bibliográficos (\texttt{*.bib}). Porém, como as regras da ABNT fazem uso quase
%abusivo da conversão para letras maiúsculas, é preciso observar o modo como se
%escreve os nomes dos autores. Na~\autoref{tabela-acentos} você encontra alguns
%exemplos das conversões mais importantes. Preste atenção especial para `ç' e `í'
%que devem estar envoltos em chaves. A regra geral é sempre usar a acentuação
%neste modo quando houver conversão para letras maiúsculas.

%\begin{table}[htbp]
%\caption{Tabela de conversão de acentuação.}
%\label{tabela-acentos}

%\begin{center}
%\begin{tabular}{ll}\hline\hline
%acento & \textsf{bibtex}\\
%à á ã & \verb+\`a+ \verb+\'a+ \verb+\~a+\\
%í & \verb+{\'\i}+\\
%ç & \verb+{\c c}+\\
%\hline\hline
%\end{tabular}
%\end{center}
%\end{table}
%%%%%%%%%%%%%%%%%%%%%%%%%%%%%%%%%%%%%%%%%%%%%%%%%%%%%%%%%%%%%%%

\subsection{Objetivos Específicos}
%-
\begin{itemize}
    \item Pesquisar e estudar o funcionamento da ferramenta OpenCV;
    \item Criar e treinar uma rede neural;
    \item Treinar a ferramenta inserindo fotos para reconhecimento facial;
    \item Treinar a ferramenta para reconhecimento de padrões de comportamento;
    \item Comunicar com protocolo MAVLink a Raspberry Pi e a Pixhawk;
    \item Realizar testes de comunicação entre a Pixhawk e a Raspberry Pi;
    \item Desenvolver um sistema de tolerância a falhas de comunicação;
    \item Pesquisar e estudar como a Raspberry Pi envia comandos de vôo;
    \item Desenvolver um algoritmo que interprete os dados extraídas da visão computacional e os converta em comandos de voo;
    \item Desenvolver um algoritmo que envie comandos de voo para o drone;
\end{itemize}
%-

\subsection{Resultados Esperados}
%-
\begin{itemize}
    \item Perfeito funcionamento da ferramenta OpenCV;
    \item Que o algoritmo de rede neural seja compilado e executado pela Raspberry Pi;
    \item Reconhecimento das pessoas inseridas no algoritmo de reconhecimento facial;
    \item Distinção entre pessoas inseridas no algoritmo de rede neural, e as que não foram inseridas;
    \item Perfeita comunicação entre a Raspberry Pi e a Pixhawk;
    \item Perfeita tolerância de falhas do sistema;
    \item Desempenho satisfatório da Raspberry Pi em enviar comandos de vôo para o drone;
    \item Comportamento de voo do drone de maneira esperada que ele desempenhe;
    \item Perfeito funcionamento do algoritmo que será desenvolvido para o funcionamento do sistema que controla e corrige em que direção o drone deve seguir;
    \item Perfeito funcionamento dos sistemas de (software e o hardware) em seguir o objeto ou pessoa que foi inserido(a) no algoritmo de reconhecimento facial.
\end{itemize}
%-
%%%%%%%%%%%%%%%%%%%%%%%%%%%%%%%%%%%%%%%%%%%%%%%%%%%%%%%%%%%%%%%%%%%
%Consulte a FAQ com perguntas frequentes e comuns no portal do \abnTeX:
%\url{https://code.google.com/p/abntex2/wiki/FAQ}.

%Inscreva-se no grupo de usuários \LaTeX:
%\url{http://groups.google.com/group/latex-br}, tire suas dúvidas e ajude
%outros usuários.
%%%%%%%%%%%%%%%%%%%%%%%%%%%%%%%%%%%%%%%%%%%%%%%%%%%%%%%%%%%%%%%%%%%

\section{Justificativa} 
%-
Devido ao aumento da criminalidade e dos altos índices de violência no cotidiano das pessoas, surgiu a cultura do medo e o sentimento de insegurança. Tais fatores demandaram algumas mudanças nos serviços de segurança patrimonial bem como, nas formas de monitoramento. Nesse sentido, tornou-se necessário expandir as formas de controle, seja por meio de câmeras de vigilância ou monitoramento eletrônico [4]. No entanto, os equipamentos atuais como as câmeras utilizadas na segurança já não mais eficientes e possuem tecnologias ultrapassadas. 
Na cidade do Rio de janeiro existe um sistema de câmeras de alta tecnologia com capacidade para realizar reconhecimento facial. Este sistema foi implantado para testes através de uma parceria entre a OI e a Huawei [5], sendo que as câmeras e a tecnologia de reconhecimento facial embarcados são de propriedade da Chinesa Huawei e possuem alto custo monetário, e o que eu quero dizer com isso é que alguns estados Brasileiros passam por uma crise financeira, porem precisam investir em segurança [6].
Este contexto de insegurança pública e falta de investimento nacional em tecnologias avançadas de segurança é que originou a motivação para a proposta deste trabalho, ou seja, a necessidade da integração entre software e hardware de baixo custo, sendo aplicados para melhorar o desempenho do sistema de segurança no Brasil, visando a vigilância, através do sensoriamento utilizando câmeras e hardware de custo acessível e softwares livres.

%-


